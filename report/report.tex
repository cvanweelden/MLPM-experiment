\documentclass[a4paper]{article}

\usepackage{amsmath}
\usepackage{url}
\usepackage{acronym}

\begin{document}

\title{MLPM project \\ The effect of smoothness}
\author{Maarten van der Velden \\ 5743087 \\ \texttt{Maarten.vanderVelden@student.uva.nl} \and Carsten van Weelden \\ 0518824 \\ \texttt{cweelden@science.uva.nl}}
\maketitle

%\begin{abstract}
%\small \textit{ }
%\end{abstract}

\acresetall

\section{Introduction}
\label{sec:introduction}

In this paper we investigate the effect of several forms of \emph{smoothness} in machine learning problems. We try regression and classification problems with artificial datasets and vary several parameters:
\begin{description}
\item[Function] The smoothness of the underlying function.
\item[Noise] The amount of noise in the dataset.
\item[Sampling] The size of the sample and the size of the domain over which it has been drawn.
\end{description}

%Which algorithms will we use?
% naive bayes (very stable because of the seperate estimates for each dimension)
% k-nn (stability depends on k)
% dec. trees or nn (unstable, dec. trees are easy to understand but difficult to analyse because of the weird decision boundaries)

\end{document}