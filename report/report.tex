\documentclass[a4paper]{article}

\usepackage{amsmath}
\usepackage{url}
\usepackage{acronym}

\acrodef{PDF}[PDF]{Probability Density Function}
\acrodef{GMM}[GMM]{Gaussian Mixture Model}
\acrodef{NB}[NB]{Naive Bayes}

\begin{document}

\title{MLPM project \\ The effect of smoothness}
\author{Maarten van der Velden \\ 5743087 \\ \texttt{Maarten.vanderVelden@student.uva.nl} \and Carsten van Weelden \\ 0518824 \\ \texttt{cweelden@science.uva.nl}}
\maketitle

%\begin{abstract}
%\small \textit{ }
%\end{abstract}

\acresetall

\section{Introduction}
\label{sec:introduction}

In this paper we investigate the effect of smoothness of the dataset on the performance of classification algorithms. In order to investigate this we generate several artificial datasets of varying smoothness and look at the accuracy and loss of the resulting classifiers. In order to get insight into the effects we decompose the loss into its bias, variance, and noise components as defined in \cite{Domingos2000}.


To investigate the effect that smoothness has on classification performance we first need to define some idea of smoothness which we can easily vary and then run a set of experiments for different levels of smoothness. We view classification as a two class problem\footnote{As contrasted with a concept learning problem, in which there is one class which needs to be distinguished from the background.}, with each class being represented by some \ac{PDF} over the attribute space. Given this view the \emph{smoothness} of a class is determined by the shape of the \ac{PDF}. We define the \ac{PDF} for each class as a \ac{GMM} with $k$ mixture components. Intuitively, the smoothness is then determined by the number of components in the mixture. With just one component the \ac{PDF} corresponds to a Gaussian distribution which is very smooth, while increasing the number of components increases the peakedness of the distribution, making it less smooth.

One way to see this is in terms of inherent noise in the problem, which we define as the noise in definition 4 of \cite{Domingos2000}:
\begin{equation}
\label{eq:noise}
N(x) = E_t[L(t,y_*)]
\end{equation}
If we keep the mean variance of the distributions equal, then with more components we will have a higher average probability that the \ac{PDF}s overlap for a datapoint $x$. Since the Bayes optimal prediction predicts the class for which the \ac{PDF} is highest at $x$, the noise is proportional to the area under the \ac{PDF} for which the \ac{PDF} of the other class is higher. In other words, the more components make up the distribution of a class, the more overlap there is between classes, and therefore the higher the noise component in the loss is.

To show this effect we generated \ac{GMM}s with an increasing amount of components and measured their noise. Averaged over 25 random models of the same number of components, it is clear that the noise increases with the number of components, growing asymptotically towards 0.5, which represents the amount of noise where the Bayes optimal decision is correct half of the time: chance level. The results are shown in Figure \ref{fig:noisepercomp}.

\begin{figure}
    \label{fig:noisepercomp}
    
\end{figure}


\section{Experimental method}
\label{sec:experimental_method}

\begin{enumerate}
\item We generate a problem containing two classes denoted $C_0$ and $C_1$. Each class is represented by a \ac{PDF} which we define as a \ac{GMM} with $k$ components\footnote{We use a 2-dimensional attribute space for ease of visualization.}. We do this for $k = [1 .. 5]$. We pick the parameters corresponding to component $j$ of class $i$ as follows. The prior probability for each component is $\pi_{ij} = rand([1 .. 5]) / Z_i$ where $Z_i$ is a normalizing factor such that $\Sigma_{j} pi_{ij} = 1$. The mean of each component is $\mu_{ij} = (rand([-1,1]),rand([-1,1]))$. We use a scalar covariance matrix $\sigma I$ with $\sigma = rand([0.1,0.4])$.
\item For each two class problem we generate a set of 100 training sets $D = \{d_1, d_2, ..., d_{100}\}$ and a corresponding test set $T$. We do this for $|D| \in {10, 100, 1000, 1000}$ with half of the data points being sampled from each class. For the test set $|T| = 1000$, also with half of the points being sampled from each class.
\item We train a classifier on each training set $d \in D$ and evaluate the accuracy on the test set.
\item Finally, we compute the average bias and average variance on $T$.
\end{enumerate}

\section{Results}
\label{sec:results}

We 




%Design choices:
% 2 class problem (2 PDFs) or concept learning problem (1 PDF versus rest). How to define the PDF (MVD, GMM, uniform)?
% Things to vary and try out: number of dimensions, real noise (next to the inherent noise), different number of components per class or number of datapoints per class, forms of PDF, learning algorithm (NB, kNN, ANN, Dec. Trees, SVM).

%Initial experiments:
% 2d 2 class
% GMM 1-5 mixtures (k) per class
% for each k:
%  m = |D| := 5 - 50 - 500 - 5000 (per class)
%  z = |T| := 1000 (500 per class)
% 100 runs each to estimate bias/variance/noise
% ?? runs each to average over different 2 class problems (e.g. pick new GMM each time)

%Workflow:
% 1. create pdfs: pick pi, mu, Sigma for each class {0,1} and k = 1-5
% 2. sample 100 datasets for each choice of k (= 400 total)
% 3. sample T
% 4. compute y_*
% 5. compute y_m = mode(t-y_*)
% 6. train classifier on each d \in D
% 7. evaluate on T
% Repeat enough times to get stable average over possible problems

%Measures:
% Accuracy: |TP|+|TN|/z
% Average bias E_x[L(y_*, y_m)]
% Average var E_x E_D[ L(y_m, y) ]
% Dataset noise L(t,y_*)

\end{document}
